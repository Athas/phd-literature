\documentclass[a4paper, oneside, final]{memoir}
\usepackage[T1]{fontenc}
\usepackage[utf8]{inputenc}
\usepackage[british]{babel}
\usepackage{mathtools}
\newcommand{\defeq}{\vcentcolon=}
\newcommand{\eqdef}{=\vcentcolon}
\pdfminorversion=4
% bedre orddeling Gør at der som minimum skal blive to tegn på linien ved
% orddeling og minimum flyttes to tegn ned på næste linie. Desværre er værdien
% anvendt af babel »12«, hvilket kan give orddelingen »h-vor«.
\renewcommand{\britishhyphenmins}{22} 

% Fix of fancyref to work with memoir. Makes references look
% nice. Redefines memoir \fref and \Fref to \refer and \Refer.
% \usepackage{refer}             %
% As we dont really have any use for \fref and \Fref we just undefine what
% memoir defined them as, so fancyref can define what it wants.
\let\fref\undefined
\let\Fref\undefined
\usepackage{fancyref} % Better reference. 

\usepackage{pdflscape} % Gør landscape-environmentet tilgængeligt
\usepackage{fixme}     % Indsæt "fixme" noter i drafts.
\usepackage{hyperref}  % Indsæter links (interne og eksterne) i PDF

\usepackage[format=hang]{caption,subfig}
\usepackage{graphicx}
\usepackage{stmaryrd}
\usepackage{amssymb}
\usepackage{amsmath}
\usepackage{listings}
\usepackage{ulem} % \sout - strike-through
\usepackage{tikz}

\renewcommand{\ttdefault}{txtt} % Bedre typewriter font
%\usepackage[sc]{mathpazo}     % Palatino font
\renewcommand{\rmdefault}{ugm} % Garamond
%\usepackage[garamond]{mathdesign}

%\overfullrule=5pt
%\setsecnumdepth{part}
\setcounter{secnumdepth}{1} % Sæt overskriftsnummereringsdybde. Disable = -1.
\chapterstyle{hangnum} % changes style of chapters, to look nice.

\makeatletter
\newenvironment{nonfloatingfigure}{
  \vskip\intextsep
  \def\@captype{figure}
  }{
  \vskip\intextsep
}

\newenvironment{nonfloatingtable}{
  \vskip\intextsep
  \def\@captype{table}
  }{
  \vskip\intextsep
}
\makeatother

\usepackage[
hyperref=auto,
backend=biber,
style=numeric,
defernumbers=true
]{biblatex}

\bibliography{bibliography}

\title{Literature Review of for my PhD}

\author{Troels Henriksen (athas@sigkill.dk)}

\date{\today}
\pagestyle{plain}

\begin{document}

\frontmatter

\maketitle
\thispagestyle{empty}

Conferences to check out:

\begin{itemize}
\item PLDI
\item PACD
\item ICS
\item PPOPP
\item CGO
\item Compiler Construction
\item FHPC
\end{itemize}

People:

\begin{itemize}
\item Cosmin Oancea
\end{itemize}

\begin{quote}
  \fullcite{Bergstrom:2013:DFN:2517327.2442525}
\end{quote}

Multicore-focused, although it looks like it would work well on GPUs
as well.  Permits several representations of ``same'' data (mostly
regular and flattened) to coincide within program.  Not too dissimilar
to Futhark's idea of representing both optimised and non-optimised
cases.  Uses coercions, that are always reversible, to change
representations, and array operations that are overloaded with respect
to representation.  Not a bad idea.  I do not understand how they can
avoid having too many expensive representation changes.  In fact,
their own dense matrix multiplication example shows that they have a
problem.

\begin{quote}
  \fullcite{Sharma:2013:DEC:2544173.2509509}
\end{quote}

Comparing the equivalence of loops at the (x86) assembly level.  Can
be used to verify validity of optimisations, but you know.  Uses test
cases to guess relationships between loop fragments, which are then
attempted proven.

\begin{quote}
  \fullcite{Wienke:2012:OFE:2402420.2402522}
\end{quote}

Uses OpenACC and compares the resulting performance (both optimised
and non-optimised) to Portland Group/PGI and hand-optimised OpenCL.
All optimisation is here is manual, by modifying the OpenMP-like
annotations of ACC.  Performance is decent and the annotation style of
programming is apparently usable (if you like that sort of thing),
although on complex benchmarks, they come quite short of matching
OpenCL performance, which is ascribed to not using device-local
memory.  One nice property is that it's fairly easy to add naive
annotations, which can then be gradually improved and refined as
necessary, given knowledge of the target hardware.

We need to get our hands on their benchmark programs.

Low level approach.

\end{document}

%%% Local Variables: 
%%% mode: latex
%%% TeX-master: t
%%% End: 
